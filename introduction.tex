\section{Introduction} 
Zuckerkandel and Pauling were arguably the first to suggest the existence of a ``molecular evolutionary clock'', based on evidence that the rate of evolutionary change of molecular sequences appeared to be very similar per unit time across diverse lineages \cite{zuckerkandl1965}. Allan Wilson was a pioneer of the application of the molecular clock and one of the first examples of a molecular phylogeny challenging palaeontological evidence came from Wilson and Sarich's paper entitled ``A molecular time scale for human evolution'' \cite{WilsonSarich1969}, in which they estimated an age of the common ancestor of humans and chimpanzees of 4-5Myr, far more recent than the figure of 20-30 Mya accepted at the time by palaeontologists.

Nevertheless a few years later Langley and Fitch \cite{LangleyFitch1974} introduced a paper on the molecular clock by saying:

\begin{quotation}
The fundamental conclusion of even the most rudimentary analysis is that amino acid sequence differences correlate well with morphological and paleontological considerations.
\end{quotation}

Evidence for the existence of a molecular clock (sometimes known as the `rate-constancy hypothesis') played a key role in support of the neutral theory of evolution and thus formed one of the battlegrounds between neutralist and selectionist views of molecular evolution \cite{Kimura1987}.

Kimura \cite{Kimura1987} put it thus:

\begin{quotation}
From the standpoint of the neutral theory, the clocklike progression of molecular evolution can be explained by assuming that the rate of production of neutral mutations per year is nearly constant among related organisms for a given molecule over time. 
\end{quotation}
 
The precise fidelity of the molecular clock was quickly brought into question by a series of empirical studies that showed that the variance of the molecular clock was greater than would be expected from Poisson error alone \cite{}  

  
  
  
  
  
  
  
  
  
  
  
  
  
  
  
  
  