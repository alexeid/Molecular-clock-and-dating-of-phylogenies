\section{Introduction} 
Zuckerkandel and Pauling were arguably the first to suggest the existence of a ``molecular evolutionary clock'', based on evidence that the rate of evolutionary change of molecular sequences appeared to be very similar per unit time across diverse lineages \cite{zuckerkandl1965}. Allan Wilson was a pioneer of the application of the molecular clock and one of the first examples of a molecular phylogeny challenging palaeontological evidence came from Wilson and Sarich's paper entitled ``A molecular time scale for human evolution'' \cite{WilsonSarich1969}, in which they estimated an age of the common ancestor of humans and chimpanzees of 4-5Myr, far more recent than the figure of 20-30 Mya accepted at the time by palaeontologists.

Nevertheless a few years later Langley and Fitch \cite{LangleyFitch1974} introduced a paper on the molecular clock by saying:

\begin{quotation}
The fundamental conclusion of even the most rudimentary analysis is that amino acid sequence differences correlate well with morphological and paleontological considerations.
\end{quotation}

Although the authors went on to show evidence that rates of nucleotide substitution are not constant across vertebrates \cite{LangleyFitch1974}, they nevertheless maintained that comparison of their phylogenetic dating of mammals against geological dates showed a remarkably good fit. 

Evidence for or against the existence of a molecular clock (sometimes known as the `rate-constancy hypothesis') provided a key battleground on which neutralist and selectionist views of molecular evolution were contested \cite{Kimura1987}.

Kimura \cite{Kimura1987} put it thus:

\begin{quotation}
From the standpoint of the neutral theory of molecular evolution, it is expected that a universally valid and exact molecular evolutionary clock would exist if, for a given molecule, the mutation rate for neutral alleles per year were exactly equal among all organisms at all times.
\end{quotation}
 
The claim of clocklike progression of molecular evolution was examined by a series of empirical studies that showed that the variance of the molecular clock was greater than would be expected from Poisson error alone \cite{LangleyFitch1974}.

  
  
  
  
  
  
  
  
  
  
  
  
  
  
  
  
  
  
  
  
  
  
  